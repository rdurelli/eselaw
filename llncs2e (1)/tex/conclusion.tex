Our long-term research goal is to find out how ADM has been applied in the literature to assist software engineers during the process of modernization of legacy system. To do so in this paper we presented a systematic review of ADM and its metamodels, following the process described by Kitchenham~\cite{Kitchenham}. Through an examination of 30 primary studies, this review has presented 22 papers which describe processes, 15 that propose a set of model transformation, 9 that present tools, 3 which present metrics and 2 that illustrate extension in the ADM's metamodels. Notice that certain primary studies were grouped in more than one category, affecting the frequency count; i.e., the sum of the frequencies shown in each facet of the Figure~\ref{map} can be greater than the total of selected studies identified, i.e., 30. %The review described in this paper reveals that the most mentioned mining techniques for crosscutting concern are Fan-In Analysis, Identifier Analysis and Dynamic Analysis. In contrast, Program Analysis Based, XScan-Concern-Peers, Data-Flow and Model Driven can be deemed as  ``evidence desert''. 

%The initial significant contributions to ADM were presented in 2007 (i.e.,~\cite{Mazon:2007:MDM:1784489.1784497, Guzman:2007:AAR:1339262.1339532, 4400179}), in 2008 had no publications. However, in 2009 we identified two paper dealing with ADM (i.e.,~\cite{delCastillo:2009:PRP:1529282.1529753, 5328801}). Next, we had a rise from 2010 (3 papers) to 2011 (8 papers). After, in 2012  
%the rate of papers gradually rose to 13. Then, it dropped out in 2013. This may happened once we carried out the review in 2013. Among these paper were found that most of them have appeared in conferences and journal, while just a was reported in workshop. We claim that researchers can use this review as a basis for advancing the field, while practitioners can use it to identify process, tools, metrics, that are well-suited to their needs. This systematic review should serve not only academic researchers but also industrial professionals, aiming at adopting some process, tool, metrics, to modernize a legacy systems within their organizations.

In summary, this review shows that there are some researchers investigating ADM and its metamodels. As consequence, we were able to identify that there are a set of processes which have been commonly used in literature to assist the software engineer to modernize legacy system by means of ADM. Afterwards, it was also possible to identify that KDM is the ADM's metamodel more utilized in the literature. %Similarly, the KDM's packages most used in the literature are Code and Action - the least used are Event and UI packages. The main future directions that emerged from this review are highlighted in Section~\ref{sec:principle_findings}. Likewise, as long term future work, related systematic reviews are about to be concluded and the relationship among their results should be investigated aiming to characterize this area in a deeper way. As soon as they are concluded, their results should be presented for the academic community as a technical report that will become available online.




%As future works, other related systematic reviews are about to be concluded and the relationship among their results should be investigated aiming to characterize this area in a deeper way. As soon as they are concluded, their results should be presented for the academic community as a technical report that will become available online.





%Research in the area of aspect-oriented modeling and model- driven code generation can result in significant advancement in development of software systems that are more maintainable, extensible and reusable. To get an overall view of the current re- search in this area, we defined a few research questions and launched a systematic mapping study. We found 65 publications that possess maximum relevance to fulfill objectives of our study. The selected papers have appeared between 2002 and 2011.
%The results of this study indicate that aspect-oriented modeling and code generation is a rather underdeveloped area. The initial significant contributions to this area were presented in 2002 (i.e. [55,56]), most papers have appeared in workshops and confer- ences, while a few were reported in journals.



%Based on the identified techniques we have extended the taxonomy proposed by Kellens et al.~\cite{Kellens}. This new taxonomy contains 7 new  mining techniques for crosscutting concerns. By using this taxonomy we hold that this taxonomy could serve as an initial roadmap to crosscutting concern researchers. Moreover, this extended taxonomy could be relevant for tool developers who might have knowledge about the best aspect indicators to use or who may have certain demands about the granularity of the results.

%The main future directions that emerged from this review are the need for empirical, comparative evaluations and the opportunity for developing combined techniques. Indeed, since every technique relies on different assumptions and uses different underlying analysis techniques, the found techniques are highly complementary, which suggests the possibility of several useful combinations. Thus, through the results obtained in this review we argue that if one pretends to devise a new mining techniques for crosscutting concerns to mine either Persistence or Observer, a good initial point is to take into consideration the combination herein illustrated in Table~\ref{table_conclusion} and~\ref{table_conclusion2} but more studies are needed because the combinations proposed did not take into consideration the versions of the system, so we intend to analyze this in future works.


% that could improve precision and recall metrics.Thus, if new mining techniques are developed then software engineers must test a wide range of concerns taking into account implementation variabilities. Finally, through the results obtained in this review we argue that if one pretends to devise a new mining techniques for crosscutting concerns to mine either Persistence or Observer, a good initial point is to take into consideration the combination herein illustrated in Table~\ref{table_conclusion} and~\ref{table_conclusion2}.


%We learned from this review that it is important to perform studies showing the pros and cons of using combinations of several mining techniques for crosscutting concern in a unique software environment. One type of analysis that can be performed is to combine techniques with high values of recall and high values of precision. Although the Table~\ref{table_conclusion} shows one tool (CBFA) using combined techniques, we can not  infer whether or not the tool has improved their capability to identify crosscutting concerns because we do not have the precision and recall metrics for each tool acting separately with the same conditions and one example is not representative. However, it could be a strategy for searching new techniques, i.e., combining CBFA with XScan. The idea behind this type of combination is to mitigate deficiencies in the way they do the identification of concerns because most probably one group of techniques is better adapted to find out a sort of concerns while others may have better results on others. Furthermore, it could be interesting to try to classify techniques according to the domain in which they act, i.e., non-functional and functional concerns. The Dynamo tool has a good precision for Persistence concern but not so good for Undo. Similarly, CBFA has a high value of precision for Undo and low value of precision for Persistence. Nevertheless, more studies are needed to confirm if Dynamo is most suitable for non-functional concern and future research might resolved this issue. We have presented candidate combinations of techniques that could improve precision and recall metrics. The comparison allowed us to show that the precision and recall can change quite a lot, when concern is different. Thus, if new mining techniques are developed then software engineers must test a wide range of concerns taking into account implementation variabilities. Finally, through the results obtained in this review we argue that if one pretends to devise a new mining techniques for crosscutting concerns to mine either Persistence or Observer, a good initial point is to take into consideration the combination herein illustrated in Table~\ref{table_conclusion} and~\ref{table_conclusion2}.