Software systems are considered legacy when their maintenance costs are raised to undesirable levels but they are still valuable for organizations. Therefore, they can not be discarded because they incorporate a lot of embodied knowledge due to years of maintenance. As these systems still provide significant business value, they must then be modernized so that their maintenance costs can be manageable and they can keep on assisting in the regular daily activities. 

%The first task that must be performed in order to carrying out a software modernization is understand the legacy system. This is not a trivial task; in fact studies estimate that between $50$ percent and $90$ percent of software maintenance involves developing an understanding of the software being maintained~\cite{Tilley95perspectiveson}, thus several approaches have been developed to support software engineers in the comprehension of systems where reverse engineering (RE) is one of them~\cite{Canfora2011}. RE supports program comprehension by using techniques that explore the source code to find relevant information related to functional and non-functional features~\cite{chikofskyTax}.

In this context, OMG has defined standards in the modernization process, creating the concept of ADM. ADM follows the MDD~\cite{5440163} guidelines and comprises three major steps. Firstly, a reverse engineering is performed starting from the source-code and a model instance Plataform-Specific Model (PSM) is created. Secondly, successive transformations are applied to this model up to reach a good abstraction level in a model called KDM (Knowledge Discovery Metamodel). Upon this model, several modernization, optimizations and modifications can be performed in order to solve problems found in the legacy system. Then, a forward engineering is carried out and the source code of the modernized target system is generated. According to the OMG the most important artifact provided by ADM is the KDM metamodel, which is a multipurpose standard metamodel that represents all aspects of the existing IT (Information  Technology) architectures. The idea behind the standard KDM is that the community starts to create parsers from different languages to KDM. As a result everything that takes KDM as input can be considered platform and language-independent. For example, a refactoring catalogue for KDM can be used for refactoring systems implemented in different languages. 

In order to  present the current state of the art and research directions in this field, we performed a systematic review of ADM. Our motivation to realize a systematic review is to identify areas that have been studied and conducted related to ADM. Although, ADM is relatively a recent approach, OMG claims that it is a standardization that aiming to join two known research field, (\textit{i}) Model-Driven Development and (\textit{ii})Software Reengineering. Thus, identifying possible gaps that have not been researched and presenting results from literature are valuable from perspective of possible future enhancements and use.  Although several studies have been reported in the context of ADM~\cite{PerezCastillo20121370, SMR:SMR582, FuentesFernandez2012247, PrezCastillo2011519}, to the best of our knowledge none systematic review has been conducted in the field of ADM. As for the fact that various types of research have appeared addressing diversifying focus areas related to the topic of modernization of legacy system by means of ADM, we claim the need for a more systematic investigation of this topic. Also this paper can help researchers in the field of modernization of legacy systems, once it provides an overview of the current state-of-the-art of the ADM. Furthermore, this paper serves as a first step towards more thorough examination of the topics addressed in it with the help of systematic literature reviews.

% If we consider Software Reengineering and Model Driven Development, the ADM approach, advocated by the OMG, is relatively recent and its first official records date is of 2003. However, ADM is a recent approach OMG claims that it is a standarlization to join Model Driven Development and Software Reengineering. Therefore, we argue that  identifying possible gaps that have not been met by ADM and try to use the works found as a reference for works of the researchers involved in this article are also objectives of this paper.



%This study also aims at identifying and presenting results from literature that are valuable from perspective of possible future enhancements and use. Although several studies have been reported in the context of ADM~\cite{PerezCastillo20121370, SMR:SMR582, FuentesFernandez2012247, PrezCastillo2011519}, to the best of our knowledge none systematic review has been conducted in the field of ADM and its metamodels. As for the fact that various types of research have appeared addressing diversifying focus areas related to the topic of modernization of legacy system by means of ADM, we claim the need for a more systematic investigation of this topic. Thus, this paper aims to present a systematic review describing research into ADM. Also we argue that this paper help researchers in the field of modernization of legacy systems, once it provides an overview of the current state-of-the-art of the ADM. Furthermore, it serves as a first step towards more thorough examination of the topics addressed in it with the help of systematic literature reviews.

This paper is structured as follows: In Section~\ref{method}, describes how the systematic review methodology has been planned, conducted and reported. In Section~\ref{sec:discussion_and_threats} there are the principle findings and the threats to validity of this study. Concluding remarks are made in Section~\ref{conclusion}.